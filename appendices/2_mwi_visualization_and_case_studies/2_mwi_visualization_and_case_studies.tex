\section{Appendix B: Many-Worlds Visualization and Case Studies}
\label{appendix:B}

\subsection{Overview of the Ripple-Based MWI Perspective}
The Many-Worlds Interpretation (MWI) proposes that every quantum measurement outcome corresponds to a distinct, non-interacting branch of the wavefunction \citep{everett1957}. Within the ripple-based framework introduced in this paper, these branches can be visualized as angular “sectors” in a polar coordinate representation, each evolving radially outward (i.e., in time) without overlapping other sectors. Orthogonality between branches naturally arises from these distinct angular domains, while normalization is preserved across all branches.

\subsection{Conceptual Alignment with MWI}
\begin{itemize}
    \item \textbf{Orthogonal Branches:}
    In the polar mapping, each branch \(\tilde{\Psi}_n\) occupies a unique angular sector. Since no two sectors overlap, these branches remain orthogonal, mirroring MWI’s principle that parallel “worlds” do not interfere after decoherence.
    \item \textbf{Normalization:}
    The radial integration of probability amplitude (scaled by \(r\,dr\,d\theta\)) ensures that the total wavefunction, summed over all angular sectors, integrates to unity. Each sector is normalized individually yet still contributes to the global total.
    \item \textbf{Phase Preservation:}
    By encoding phase as hue, the framework preserves coherence within a branch. Interference phenomena, vital to quantum mechanics, thus remain accurately depicted within each “world.”
\end{itemize}

\noindent
\textit{[Optional: Insert a simplified schematic or flowchart illustrating how separate angular sectors map to distinct MWI branches.]}

%------------------------------------------------
% FIGURE PLACEHOLDER: MWI conceptual figure
% e.g., \begin{figure}[h]
%       \centering
%       \includegraphics[width=0.6\textwidth]{figures/MWI_sectors.png}
%       \caption{Schematic showing how the ripple-based framework assigns orthogonal angular sectors to different Many-Worlds branches.}
%       \label{fig:MWI_sectors}
%       \end{figure}
%------------------------------------------------


\subsection{Minkowski Diagram of the Framework}
To illustrate the relativistic underpinnings, we provide a Minkowski diagram highlighting the light-cone structure, selected spacetime events, and how the ripple model (treating \(r = c\,t\)) captures time-like evolution. This diagram also shows how each angular sector fans out from the origin, representing an evolving “world” in MWI.

%------------------------------------------------
% FIGURE PLACEHOLDER: Minkowski diagram
% e.g., \begin{figure}[h]
%       \centering
%       \includegraphics[width=\textwidth]{figures/minkowski_diagram.png}
%       \caption{Minkowski spacetime diagram showing the light cone, selected spacetime events, and the polar ripple model (time as radius).}
%       \label{fig:minkowski_diagram}
%       \end{figure}
%------------------------------------------------

\subsection{Simulation and Visualization Examples}

In this section, we showcase representative simulations that highlight the practical utility of the ripple-based approach. All scripts and corresponding outputs are referenced here but stored in the supplementary GitHub repository for brevity.

\subsubsection{Single Branch Simulation}
\paragraph{Goal:} Demonstrate normalization and phase coherence within a single branch.
\begin{itemize}
    \item \textbf{Setup:} A single Gaussian wavefunction, \(\Psi(x,t)\), is mapped to an angular sector \([0^\circ,180^\circ)\) in polar coordinates.
    \item \textbf{Outcome:} Visual inspection confirms that probability integrates to unity, with the hue-based phase representation capturing minor interference or dispersion effects.
\end{itemize}

\noindent
\textit{[Optional: Insert figure(s) showing snapshots of the single-branch wavefunction at various times, highlighting amplitude and hue.]}

%------------------------------------------------
% FIGURE PLACEHOLDER: Single-branch wavefunction
%------------------------------------------------

\subsubsection{Multiple Branches Simulation}
\paragraph{Goal:} Illustrate orthogonality and independent evolution of multiple branches.
\begin{itemize}
    \item \textbf{Setup:} Two or more wavefunctions, each assigned a distinct angular range \(\Theta_n\). For instance, \(\Psi_1\) in \([0^\circ,120^\circ)\) and \(\Psi_2\) in \([120^\circ,240^\circ)\).
    \item \textbf{Outcome:} Branches do not overlap in \(\theta\)-space, so they remain orthogonal. Each sector evolves radially, preserving the global normalization without interference between branches.
\end{itemize}

\noindent
\textit{[Optional: Insert figure(s) showing multi-branch wavefunctions, emphasizing how each sector evolves independently.]}

%------------------------------------------------
% FIGURE PLACEHOLDER: Multi-branch wavefunction
%------------------------------------------------


\subsubsection{Infinite Branching Limit}
\paragraph{Goal:} Demonstrate scalability as the number of branches \(N\to\infty\).
\begin{itemize}
    \item \textbf{Setup:} Model a scenario where a single wavefunction splits into many possible outcomes—e.g., repeated measurements or multi-slit expansions—gradually filling more angular sectors.
    \item \textbf{Outcome:} As the sector width decreases (\(\Delta\theta \approx \tfrac{2\pi}{N}\)), the representation approaches a continuum, visually resembling a “spray” of possible evolutions. Yet each sector remains individually normalized and maintains a distinct phase evolution.
\end{itemize}

\noindent
\textit{[Optional: Insert figure/animation illustrating a large number of angular sectors, providing a near-continuous distribution of outcomes.]}

%------------------------------------------------
% FIGURE PLACEHOLDER: Infinite branching limit
%------------------------------------------------

\subsubsection{Interference Patterns and Case Studies}
We now focus on the specific quantum scenarios that highlight the framework’s interpretive strength:

\paragraph{1) Klein--Gordon and Dirac Gaussian Packets}
\begin{itemize}
    \item \textit{Stationary vs.\ Moving Packets:} Reveal relativistic dispersion, with radial “ripples” indicating time evolution and hue changes showing phase shifts.
    \item \textit{Commentary:} Contrasting the stationary with the moving case underscores how the Lorentz factor emerges geometrically in Minkowski space, indirectly modifying the wavefunction shape.
\end{itemize}

\paragraph{2) Superposition and Entanglement}
\begin{itemize}
    \item \textit{Constructive/Destructive Interference:} Visual cues (amplitude/hue) pinpoint when phases align or oppose each other.
    \item \textit{Entangled Subsystems:} Phase correlations are readily apparent when mapped into polar form, aiding intuition about coherence across spatial separation.
\end{itemize}

\paragraph{3) Double-Slit Experiment}
\begin{itemize}
    \item \textit{Interference Fringes:} Appear as concentric arcs or patterns in the radial dimension (time), with hue transitions reflecting phase relationships.
    \item \textit{Comparison to Traditional 2D Plots:} The polar representation can consolidate amplitude and phase in a single graphic, revealing interference far more explicitly than separate intensity and phase plots.
\end{itemize}

\paragraph{4) Tunneling and Scattering Scenarios}
\begin{itemize}
    \item \textit{Barrier Penetration:} Partial reflection (phase-shifted wave traveling backward in \(\theta\)) and transmitted wave appear as distinct ripples, each with characteristic hue shifts.
    \item \textit{Energy Dependence:} Higher-energy packets exhibit subtler differences in reflection/transmission, readily captured by the radial phase evolution.
\end{itemize}

\noindent
\textit{[Optional: Insert relevant figures/animations for each scenario, alongside short captions explaining the observed patterns.]}


\subsection{Interpretation and Comparison with Conventional Methods}
The examples above illustrate the advantages of visualizing quantum phenomena in a polar coordinate system aligned with Minkowski geometry:
\begin{itemize}
    \item \textbf{Amplitude \& Phase in One Plot:} Unlike typical 2D Cartesian plots (e.g., \(\mathrm{Re}(\Psi), \mathrm{Im}(\Psi)\), or intensity alone), the ripple framework encodes both amplitude (radius/brilliance) and phase (hue) simultaneously.
    \item \textbf{Clear MWI Branching:} Assigning separate angular sectors to each branch naturally enforces orthogonality and visually demonstrates the idea of “non-interacting worlds.”
    \item \textbf{Immediate Interference Recognition:} Oscillations in amplitude and abrupt hue changes indicate constructive or destructive interference without requiring multiple subplots.
\end{itemize}

\noindent Ultimately, the \emph{ripple-based} approach offers a vivid, relativistically consistent visualization that captures the essence of MWI and advanced quantum phenomena in a single, unified representation.

\bigskip
\noindent
\textbf{Note on Figures and Animations:}\\
All simulations referenced herein, along with the scripts used to generate them, are located in the project’s public GitHub repository. Readers are encouraged to consult these files for high-resolution images, dynamic animations, and additional context.

\newpage
