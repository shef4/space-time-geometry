
\section{Appendix A: Detailed Mathematical Derivations}
\label{appendix:A}

\subsection{Coordinate Transformation Justification}
The transition from Cartesian to polar (or cylindrical) coordinates underpins our approach of interpreting time as a spatial dimension. By setting \( r = c\,t \), time is scaled by the speed of light to maintain consistent length units. This choice resonates with relativistic principles, where space and time are woven into a unified Minkowski framework.

\subsubsection{Scaling Factor Selection}
The factor \(\sqrt{\frac{180^\circ}{L}}\) ensures wavefunction normalization after the coordinate transformation:
\[
\int_{0}^{2\pi} \int_{0}^{c\,t} \bigl|\tilde{\Psi}(r,\theta)\bigr|^2 \,r\,dr\,d\theta \;=\; 1.
\]
This constant preserves the total probability of the system, thereby safeguarding the physical validity of the quantum state.

\subsection{Normalization Preservation}
To rigorously demonstrate that normalization is maintained under the polar transformation, consider:
\[
\tilde{\Psi}(r, \theta) \;=\;
\begin{cases}
\sqrt{\frac{180^\circ}{L}} \,\Psi\bigl(x(\theta), t\bigr), & 0^\circ \le \theta < 180^\circ, \\[6pt]
\sqrt{\frac{180^\circ}{L}} \,\Psi^*\bigl(x(\theta - 180^\circ), t\bigr), & 180^\circ \le \theta < 360^\circ,
\end{cases}
\]
where \(r = c\,t\). Thus,
\[
\int_{0}^{2\pi} \!\int_{0}^{c\,t} \bigl|\tilde{\Psi}(r,\theta)\bigr|^2 \,r\,dr\,d\theta
\;=\;
\frac{180^\circ}{L}
\int_{0}^{2\pi} \!\int_{0}^{c\,t}
\bigl|\Psi\bigl(x(\theta), t\bigr)\bigr|^2 \,r\,dr\,d\theta.
\]
Using \(x = x(\theta)\) and \(dx = \tfrac{L}{180^\circ}\,d\theta\), the integral over \(\theta\) recovers the original Cartesian normalization:
\[
\frac{180^\circ}{L} \,\cdot\,\frac{L}{180^\circ}
\int_{-L/2}^{L/2} \!\int_{0}^{c\,t}
\bigl|\Psi(x,t)\bigr|^2 \,c\,dt\,dx
\;=\;
\int_{-L/2}^{L/2} \!\bigl|\Psi(x,t)\bigr|^2\,dx
\;=\; 1,
\]
assuming \(\Psi(x,t)\) is normalized in its original Cartesian form.

\subsection{Phase Encoding and Interference}
Representing the wavefunction’s phase \(\phi\) via hue is vital for visualizing quantum interference:
\[
\text{Hue}(\phi) \;=\; \frac{\phi + \pi}{2\pi}, 
\quad \phi \in [-\pi, \pi].
\]
This mapping linearly translates phase angles to a continuous color spectrum, allowing constructive (\(\Delta \phi \approx 0\)) and destructive (\(\Delta \phi \approx \pi\)) interference to be readily discernible.

\subsubsection{Interference Patterns}
For a superposition of two wavefunctions, \(\Psi = \Psi_1 + \Psi_2\), with phases \(\phi_1\) and \(\phi_2\), the probability density is
\[
\bigl|\Psi\bigr|^2
\;=\;
\bigl|\Psi_1\bigr|^2
\;+\;
\bigl|\Psi_2\bigr|^2
\;+\;
2\,\bigl|\Psi_1\bigr|\bigl|\Psi_2\bigr|\cos\bigl(\phi_1 - \phi_2\bigr).
\]
The term \(\cos(\phi_1 - \phi_2)\) governs the interference pattern, which appears as variations in both amplitude and hue.

\subsection{Normalization and Orthogonality in MWI}
Within the Many-Worlds Interpretation (MWI), each branch \(n\) is orthogonal to every other branch \(m\):
\[
\int_{0}^{2\pi}
\tilde{\Psi}_n^*(r,\theta)\,\tilde{\Psi}_m(r,\theta)\,d\theta
\;=\; 0,
\quad n \;\neq\; m.
\]
Distinct, non-overlapping angular sectors maintain orthogonality by allocating each branch to a unique segment of \(\theta\). The ripple-based framework preserves total probability across all branches via normalized wavefunctions in each sector.

\subsection{Energy Conservation}
For a Klein--Gordon wavefunction \(\Psi(x,t)\) obeying
\[
\Box\,\Psi
\;-\;
\frac{m^2 c^2}{\hbar^2}\,\Psi
\;=\; 0,
\]
the energy density is
\[
\mathcal{E}
\;=\;
\hbar^2\bigl|\partial_t \Psi\bigr|^2
\;+\;
\hbar^2\,c^2\,\bigl|\partial_x \Psi\bigr|^2
\;+\;
m^2\,c^4\,\bigl|\Psi\bigr|^2.
\]
Integrating \(\mathcal{E}\) over all space yields a constant total energy
\[
E
\;=\;
\int_{-L/2}^{L/2}
\mathcal{E}\,dx
\;=\;
\text{const.},
\]
which the ripple-based framework preserves due to its consistent normalization. Thus, energy remains conserved under the polar transformation, confirming physical fidelity for relativistic wavefunctions.


\newpage
